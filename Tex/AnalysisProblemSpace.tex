% -*- root: ../main.tex -*-
\chapter{Analisi dello spazio del problema}
	\section{Knowledge crunching}
    Per poter fare un'analisi soddisfacente è fondamentale avere una buona conoscenza del \textbf{dominio} e della sua \textbf{terminologia}, cosa che, se fatta in autonomia, potrebbe richiedere parecchio tempo. Per questo motivo è importante \textbf{collaborare} con gli \textbf{esperti di dominio}, che aiutano a comprendere i componenti del dominio, le dinamiche, i termini tecnici e gli aspetti critici.
    Il \textbf{knowledge crunching} è il percorso che porta chi deve realizzare il sistema ad avere una \textbf{migliore comprensione} non solo di \textbf{quanto} deve essere fatto, ma del \textbf{perché}, permettendo di non fermarsi alle \textbf{specifiche}, ma di capirne il \textbf{senso}. 
    
    \subsection{Sessioni}
	Nel nostro caso le sessioni di \textbf{knowledge crunching} sono state svolte interpellando un \textbf{esperto di dominio} per via telematica.
	\begin{itemize}
	    \item \textbf{Prima sessione:} nella prima sessione, il \textbf{colloquio} con l'esperto ha permesso al team di comprendere:
	    \begin{itemize}
	        \item gli aspetti principali della \textbf{gestione} del canile;
	        \item le \textbf{figure} coinvolte;
	        \item i \textbf{ruoli} presenti all'interno del canile;
	        \item le \textbf{mansioni} che vengono svolte quotidianamente;
	        \item i \textbf{processi di gestione}.
	    \end{itemize}
	    Queste conoscenze hanno permesso di dare il \textbf{kick-off} al progetto.
	    %Organization Chart - Organigramma
        \begin{figure}[ht]
            \caption{Organization Chart}
            \centering
            \includegraphics[width=1\textwidth]{DrawIo/organizationChart.png}
        \end{figure}
	    \item \textbf{Sessioni successive:} successivamente il processo è stato continuo, sono stati svolti degli \textbf{incontri periodici} per:
	    \begin{itemize}
	        \item controllare che il progetto proseguisse nella direzione sperata;
	        \item \textbf{sciogliere} alcuni \textbf{dubbi} sorti in corso d'opera;
	        \item \textbf{verificare} la corretta \textbf{comprensione} di alcuni concetti chiave. 
	    \end{itemize}
	\end{itemize}
    
    \subsection{Svolgimento}
    \subsubsection{Comunicazione}
	Data l'ovvia problematicità nell'incontrarsi, la comunicazione è stata gestita da remoto, utilizzando le seguenti piattaforme:
	\begin{itemize}
	    \item \textbf{Telegram}, per chiamate vocali e brevi chiarimenti
	    \item \textbf{Discord}, per la condivisione schermo con gli artefatti visuali
	\end{itemize}
	Durante le interazioni sono state create multiple \textbf{rappresentazioni visuali} a guida e supporto della conversazione. 
	
	\subsubsection{Percorso}
	\begin{itemize}
	    \item Il gruppo parlando con il \textbf{responsabile del canile} (il maggiore esperto del dominio) ha \textbf{indagato} con maggiore enfasi i concetti necessari per la creazione di un \textbf{buon software}, tralasciando dettagli \textbf{pleonastici}. 
	    \item Il \textbf{focus} è stato su capire quello che serve per lo sviluppo di un software che soddisfi le \textbf{reali necessità} del cliente. I dettagli su cui l'esperto del dominio ricadeva spesso, infatti, hanno rivelato dove le energie dovessero essere maggiormente spese. 
	    \item In questo caso gli \textbf{esperti} del dominio sono anche i \textbf{clienti}. Questi ultimi forniscono il \textbf{problem-space}, mentre i primi forniscono il \textbf{solution-space}.
	    \end{itemize}
	
	Per la comprensione degli obbiettivi di più alto business value, si è costruita in maniera interattiva e collaborativa un'impact map, sotto la supervisione di team, scrum master e product-owner. L'uso di questo diagramma mira a mantenere un livello non troppo formale da risultare incomprensibile agli estranei al dominio, ma nemmeno totalmente informale, perdendo efficacia comunicativa. 
	
	%IMPACT MAP DIAGRAM
    \begin{figure}[ht]
        \caption{Impact-map obbiettivi business}
        \centering
        \includegraphics[width=1\textwidth]{DrawIo/impactMap.png}
    \end{figure}

    Il risultato ha portato in evidenza i due principali obbiettivi di business, evidenziando in maniera chiara la loro suddivisione, e le soluzioni per raggiungere i risulati attesi.
    
    Con una visone d'insieme leggermente più definita, ci siamo cimentati nella stesura delle \textbf{user stories} per ogni ruolo. Dalla moltitudine emersa, sono state identificate, cercando di seguire i requisiti di business individuati,  le più calzanti, legate ad ogni obiettivo dell'\textbf{impact map}.

	La definizione delle user-stories ha portato naturalmente a indagare il significato di alcuni termini usati. Questo ha spinto subito allo sviluppo concorrente di un ubiquitous language per i vocaboli più importanti. Questo è stato poi continuamente raffinato durante i successivi meeting. 
	Nel prossimo capitolo viene analizzato questo processo. 
    
	\section{Definizione Ubiquitus language}	
	L'Ubiquitous Language deve essere espresso nel modello di dominio, infatti unisce le persone del team di progetto.
    Lo scopo è eliminare le imprecisioni e le contraddizioni degli esperti di dominio, non è infatti imposto da questi, ma raggiunto collaborativamente.
    L'Ubiquitous Language si evolve nel tempo, non è definito interamente in una sola riunione, i concetti spesso si aggiungono, vengono sviscerati e partecipano nella comprensione del dominio. Infatti quelli che non fanno parte dell'Ubiquitous Language devono essere rifiutati.
    
    %TABELLA UBIQUITOUS LANGUAGE
    \begin{figure}[ht]
        \caption{Tabella ubiquitous language}
        \centering
        \includegraphics[width=1\textwidth]{DrawIo/ubiquitousLanguage.png}
    \end{figure}
