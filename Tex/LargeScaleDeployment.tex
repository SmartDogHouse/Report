% -*- root: ../main.tex -*-

% In questa sezione va discussa, eventualmente con l'ausilio di opportuni diagrammi (componenti, deployment), l'evoluzione del progetto presentato immaginando che venga adottato su larga scala. I dettagli qui esposti devono quindi astrarre dalle specifiche dell'elaborato qualora l'implementazione sia stata focalizzata su uno scenario isolato. A titolo d'esempio, qualora applicabile, devono essere evidenziate le criticità che si potrebbero incontrare e devono essere proposte soluzioni tipiche in contesti di cloud architecture per garantire un'adeguata resilienza, in termini di availability e scalability del sistema.
% 6000 - 12000 battute

\chapter{Analisi di Deployment su Larga Scala}
L'applicativo è stato modellato in ogni sua parte per scalare facilmente all'occorrenza. Il design distribuito e in cloud ne facilita l'eventuale evoluzione su larga scala.
Verranno di seguito brevemente analizzati due scenari: 
\begin{itemize}
    \item  \textbf{Evoluzione Interna} In termini di scalabilità del singolo software lo scenario più realistico consiste nell'aggiunta di elementi (gabbie o animali) da parte di un grande canile. L'aggiunta, anche cospicua, di elementi non porterebbe ulteriore complessità al sistema. Gli unici requisiti riguardano comprare l'hardware specifico da installare. Questo si occuperà attraverso il programma di mandare i dati direttamente al cloud, il quale li processerà e mostrerà le informazioni aggiornate sull'applicativo.
    \item  \textbf{Diffusione su Larga Scala} In termini di diffusione del software in molteplici copie lo scenario riguarderebbe l'adozione dell' applicativo da parte di molteplici amministrazioni. Grazie all'adozione dei servizi AWS in Cloud, l' \textbf{availability} e la \textbf{scalability} del sistema sono garantite anche ad alti livelli.
\end{itemize}