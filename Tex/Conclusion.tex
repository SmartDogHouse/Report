% -*- root: ../main.tex -*-

% Esporre brevemente le considerazioni conclusive sul progetto presentato, indicando anche i possibili sviluppi futuri.
% 1500 - 3000 battute

\chapter{Conclusioni}
Al termine del percorso che ha portato alla realizzazione di questo progetto è stato raggiunto un risultato che riteniamo molto soddisfacente. Sono state utilizzate e integrate diverse tecnologie, il che ci ha permesso di acquisire ed affinare competenze a tutto tondo. Riteniamo inoltre che il progetto sia stato particolarmente ambizioso e gli sforzi sono stati commisurati alla soddisfazione finale per averlo completato. Questo lavoro, inoltre, potrebbe rappresentare una buona base di partenza per lo sviluppo di un sistema reale che porti un reale vantaggio alla realtà dei rifugi per animali. 
    \section{Sviluppi Futuri}
    In futuro sarebbe possibile estendere il sistema aggiungendo una parte gestionale. I due canili che abbiamo interpellato, infatti, sono sprovvisti anche della più basilare forma di sistema informatico, fatta eccezione per l'anagrafe canina che però non è ottimizzata per l'utilizzo che ne devono fare e ancora tanto lavoro viene affidato alle capacità di ragionamento e deduzione del personale del canile. 
    Si potrebbe inoltre integrare il sistema aggiungendo altre feature, come l'inserimento di un nuovo cane mediante la lettura dei dati direttamente dal chip oppure l'introduzione di una più importante elaborazione dei dati, soprattutto per quanto riguarda le immagini provenienti dalla videocamera che potrebbero dare un importante aiuto ai gestori se gli venisse applicata una logica che, ad esempio, permette di rilevare comportamenti anomali del cane. Si potrebbe inoltre aggiungere un sistema di tracciamento mediante gps in maniera tale da permettere ai volontari o alle famiglie che, nella fase di pre-affidamento, vorrebbero abituare il cane alla loro presenza, di portare il cane a fare una passeggiata, senza temere che il cane possa scappare e perdersi.
    
    Un altro possibile sviluppo riguarda il dispiegamento del sistema su larga scala, affinché possa essere utilizzato da più canili. A questo punto se si riuscisse ad introdurre un'elaborazione migliore delle soglie standard dei consumi e dei parametri vitali, ci si potrebbe basare sulle medie di più cani, cosa che permetterebbe di evitare che debba essere il veterinario ad impostare manualmente le soglie che ritiene più adeguate per ciascuna cane basandosi su razza, tagli, età e precedenti clinici.
