% -*- root: ../main.tex -*-
\chapter{Iterazioni}

\section{Sprint 0}
All'interno dello sprint 0 il focus è stato sullo scegliere le giuste tecnologie di sviluppo e sull'apprenderle in maniera sufficiente per il kick-off del progetto.

\paragraph{Deliverables} al termine di questo sprint si sono acquisite le competenze di base per poter iniziare a lavorare con AWS e i dispositivi IoT. Inoltre ci si è portati avanti con lo scheletro del sito.
\begin{itemize}
    \item ambiente di lavoro Linux standardizzato e virtualizzato
    \item sito linkato, con le prime due pagine base: home e login
    \item ESP32 con firmware installato MicroPython, script e guida all'uso
    \item codice per sensori base, test relativi e stubs per eseguirli senza necessità del micro-controllore
    \item repository software con CI e test automatizzati 
    \item database progettato e popolato con qualche dato ai fini di testing
    
\end{itemize}

\section{Sprint 1}
    \subsection{Pianificazione}
        \subsubsection{Parte 1} 
            raffina e rivedi i prod. backlog, scelta sprint goal (what)
        \subsubsection{Parte 2}
            si decidono gli item (how), solo team senza prod owner
    \subsection{Daily scrum}
    \subsection{Meeting finale}
        \subsubsection{Product backlog refinement}
        \subsubsection{Sprint review}
        \subsubsection{Retrospettiva}

    
\section{Sprint 2}


\section{Sprint 3}


\section{Sprint 4}


