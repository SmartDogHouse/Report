% -*- root: ../main.tex -*-
\chapter{Sprints}

\section{Svolgimento}
Gli sprint sono stati portati avanti nel seguente modo:
    \paragraph{Sprint Planning}
        Pianificazione a inizio sprint degli obiettivi, tempistiche e responsabilità nel periodo dello sprint corrente. Diviso in due parti:
        \begin{itemize}
        \item\textbf{parte 1} 
            Viene raffinato e rivisto il product backlog, viene effettuata la scelta dello sprint goal (what).
        \item\textbf{parte 2}
            Si decidono gli item e viene raffinato come implementarli (how). Effettuato con solo il team senza la figura del product owner
        \end{itemize}
    \paragraph{[Iterativo] Daily scrum} Breve meeting svolto giornalmente. Viene utilizzato per gli aggiornamenti sull'andamento del progetto, senza scendere nei dettagli implementativi.
    \paragraph{[Occasionale] Pair Programming } Utilizzato per risolvere problemi che causano il blocco di un componente del team per parecchio tempo su una issue.
    \paragraph{Meeting finale}
        Riflessioni e considerazioni finali sullo spint passato. Suggerimenti per migliorare il prossimo. Diviso in tre parti: 
        \begin{itemize}
        \item\textbf{Product backlog refinement} aggiunta di dettagli e riordino del product backlog
        \item\textbf{Sprint review} è stato ispezionato l'incremento, il Minimum Viable Product o di risultati sul processo. Discernere cosa è stato fatto e cosa no
        \item\textbf{Retrospettiva} Considerazioni sul team stesso e sui miglioramenti per il prossimo sprint. 
        \end{itemize}
        
        

\section{Sprint 0}
All'interno dello sprint 0 il focus è stato sullo scegliere le giuste tecnologie di sviluppo e sull'apprenderle in maniera sufficiente per il kick-off del progetto.
L'obiettivo è stato raggiungere una conoscenza e competenza minima per sviluppare il design in maniera consapevole e ottimale.
\paragraph{Deliverables} 
al termine di questo sprint si sono acquisite le competenze di base per poter iniziare a lavorare con AWS e i dispositivi IoT. Inoltre ci si è portati avanti con lo scheletro del sito.
\begin{itemize}
    \item ambiente di lavoro Linux standardizzato e virtualizzato
    \item sito linkato, con le prime due pagine base: home e login
    \item ESP32 con firmware installato MicroPython, script e guida all'uso
    \item codice per sensori base, test relativi e stubs per eseguirli senza necessità del micro-controllore
    \item repository software con CI e test automatizzati 
    \item database progettato e popolato con qualche dato ai fini di testing
\end{itemize}

\section{Sprint 1}
All'interno dello sprint 1 il focus è stato sull'usare le competenze tecnologiche acquisite precedentemente per sviluppare i componenti principali del progetto. Sono stati scelti come obiettivi le user stories per l'automatizzazione di cibo e acqua e la visualizzazione delle informazioni relative a un animale sulla webpage.
\paragraph{Deliverables}
al termine di questo sprint sono state implementate le funzioni base dei maggiori componenti per l'automatizzazione fisica di cibo e acqua e il relativo prototipo fisico. Sono state aggiunte sull'applicativo la vista delle informazioni dell'animale e l'impostazione del cibo da erogargli. 
\begin{itemize}
    \item codice per i sensori/attuatori per acqua e cibo, con stubs, test e automatizzazione. (Livello acqua, elettrovalvola, motore, bilancia, laser e rilevatore di luce)
    \item prototipo fisico per i sensori per acqua e cibo. 
    \item database migliorato e rifattorizzato
    \item visualizzazione dei dati del cane sul sito 
    \item visualizzazione dei grafici delle statistiche sul sito 
\end{itemize}

\section{Sprint 2}
Nello sprint 2 il focus è stato sul realizzare la logica d'orchestrazione della sensoristica e l'integrazione tra essa e l'applicativo per mezzo del protocollo MQTT. Sono stati prodotti gli artefatti di documentazione e schemi esemplificativi. Inoltre, è stato dato peso alla creazione delle query che permettono di integrare le funzioni del sito con il database. In particolare ci si è concentrati sulle user-stories prioritarie, che costituiscono il core del dominio.
\paragraph{Deliverables}
\begin{itemize}
    \item sviluppate query necessarie alla logica centrale dell'applicazione web
    \item grafici per la temperatura e l'umidità nell'applicazione web
    \item implementata coverage dei test Python, con relativa pagina web su Github pages. 
    \item aggiunta Quality Assurance con SonarCloud, Codacy e Codefactor
    \item implementata Continuous Delivery con GitHub Actions e GitHub Releases
    \item implementata macchina a stati per food delivery e water automation
\end{itemize}

\section{Sprint 3}
Nello sprint 3 sono state implementate le query che verranno chiamate dal sito per mostrare gli elementi all'utente. Inoltre, è stato migrato il codice JavaScript in un repository dedicato che permette il deployment delle funzioni in automatico su AWS con SAM. Sono state sviluppate le funzioni per ottenere i dati della sensoristica del collare smart, implementando la logica per il controllo dei valori e l'invio delle notifiche su MQTT.
\paragraph{Deliverables}
\begin{itemize}
    \item sviluppate query complesse necessarie alla logica centrale dell'applicazione web
    \item repository dedicato delle funzioni JavaScript per AWS
    \item setup framework SAM per automazione AWS
    \item prototipo fisico per i sensori di temperatura e battito
    \item implementata macchina a stati per smart-collar heartbeat e temperature
\end{itemize}

\section{Sprint 4}
Lo sprint 4 è stato il più impegnativo, perché vi era la necessità di chiudere il progetto, portando a termine tutti i task preventivati.
Una grossa parte dello sprint è stata dedicata all'integrazione delle diverse parti, alla chiusura delle ultime interrogazioni al database, all'inserimento degli ultimi pezzi dell'interfaccia grafica che permettono all'utente di interagire con il back-end e alla conclusione del sistema di sorveglianza.
\paragraph{Deliverables}
\begin{itemize}
    \item Completamento WebApp che permette all'utente di interagire con il sistema, visualizzando i dati e apportando delle modifiche.
    \item Completamento Back-end che gestisce la logica del sito
    \item Completamento gestione di sensori, attuatori e sistema di sorveglianza.
\end{itemize}

\section{Sprint 5}
Lo sprint 5 è stato lo sprint finale di revisione progetto, per assicurare la massima qualità e rifinire gli ultimi dettagli.
In questo sprint si sono corretti piccoli bug, sono state apportate migliorie estetiche e funzionali. Infine anche la documentazione e il report sono stati revisionati.
\paragraph{Deliverables}
\begin{itemize}
    \item Migliorie grafiche.
    \item Migliorie al codice.
    \item Video del prototipo. 
\end{itemize}


