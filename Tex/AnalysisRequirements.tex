
\chapter{Analisi dei Requisiti}
In questa fase sono stati individuati i \textbf{requisiti del sistema}, partendo dalle descrizioni di alto livello, ottenute dal committente durante il \textbf{knowledge krunching}. Successivamente si è proceduto con un raffinamento che ha portato alla definizione di requisiti più \textbf{specifici},\textbf{chiari} e \textbf{strutturati}.
	\section{Requisiti di Business}
	Si definiscono di seguito le aspettative del cliente e i requisiti che il prodotto dovrà soddisfare, espressi con una terminologia ad elevato livello astrattivo.
        \begin{itemize}
            \item Il prodotto dovrà diminuire l'intervento umano necessario per la quotidiana cura degli ospiti (Riempimento ciotola acqua/cibo); la riduzione del lavoro deve essere maggiore o uguale al 30\%.
            \item Il prodotto dovrà consentire di diminuire il lavoro su base volontaria, grazie alla riduzione delle operazioni di cura quotidiana, permettendo ai volontari di concentrarsi solo sulla socializzazione e lo svago dell'animale.
            Diminuisce così l'interferenza al di fuori delle loro mansioni, e migliora la tracciabilità del lavoro svolto, essendo il lavoro volontario incostante e meno affidabile.
            \item Opzionalmente il prodotto dovrà fornire un monitoraggio a distanza degli ospiti.
            Questo consentirà al personale incaricato di monitorare parametri come: frequenza cardiaca e temperatura corporea. Verrà diminuito anche l'intervento veterinario, non essendo necessaria la presenza in loco del professionista. La riduzione delle presenze dovrà consentire di diminuire i costi di un valore maggiore o uguale al 10\%.
        \end{itemize}
	
	\section{Requisiti Utente}
	Il prodotto dovrà consentire all'utente di svolgere le seguenti azioni:
	   \begin{itemize}
            \item %TODO
        \end{itemize}
        \subsection{User stories}

	    \subsection{Casi d'uso}
	    
	\section{Requisiti Funzionali}
	   \begin{itemize}
            \item Applicativo web
            \item Applicazione video
            \item Opzionale collare
            \item Strumentazione gabbia
        \end{itemize}
	\section{Requisiti non Funzionali}
	\section{Requisiti Implementativi}
