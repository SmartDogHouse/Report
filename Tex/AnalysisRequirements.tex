
\chapter{Analisi dei Requisiti}
In questa fase sono stati individuati i \textbf{requisiti del sistema}, partendo dalle descrizioni di alto livello, ottenute dal committente durante il \textbf{knowledge krunching}. Successivamente si è proceduto con un raffinamento che ha portato alla definizione di requisiti più \textbf{specifici}, \textbf{chiari} e \textbf{strutturati}.
	\section{Requisiti di Business}
	Si definiscono di seguito le aspettative del cliente e i requisiti che il prodotto dovrà soddisfare, espressi con una terminologia ad elevato livello astrattivo.
        \begin{itemize}
            \item Il prodotto dovrà diminuire l'intervento umano necessario per la quotidiana cura degli ospiti (Riempimento ciotola acqua/cibo); la riduzione del lavoro deve essere maggiore o uguale al 30\%.
            \item Il prodotto dovrà consentire di diminuire il lavoro su base volontaria, grazie alla riduzione delle operazioni di cura quotidiana, permettendo ai volontari di concentrarsi solo sulla socializzazione e lo svago dell'animale.
            Diminuisce così l'interferenza al di fuori delle loro mansioni, e migliora la tracciabilità del lavoro svolto, essendo il lavoro volontario incostante e meno affidabile.
            \item Opzionalmente il prodotto dovrà fornire un monitoraggio a distanza degli ospiti.
            Questo consentirà al personale incaricato di monitorare parametri come: frequenza cardiaca e temperatura corporea. Verrà diminuito anche l'intervento veterinario, non essendo necessaria la presenza in loco del professionista. La riduzione delle presenze dovrà consentire di diminuire i costi di un valore maggiore o uguale al 10\%.
        \end{itemize}
	
	\section{Requisiti Utente}
	Il prodotto dovrà consentire all'utente di svolgere le seguenti azioni:
	   \begin{itemize}
            \item %TODO
        \end{itemize}
        \subsection{User stories}
        \begin{itemize}
            \item Gestore
            \begin{enumerate}
                \item 
            \end{enumerate}
            \item Come \textbf{Veterinario} voglio :
            
            \item Addetti
            \begin{enumerate}
                \item 
            \end{enumerate}
        \end{itemize}
	    \subsection{Casi d'uso}
	    
	\section{Requisiti Funzionali} %obbligatori, desiderabili e opzionali
	   \begin{itemize}
            \item Applicativo web 
            \item Applicazione video
            \item Opzionale collare
            \item Strumentazione gabbia
        \end{itemize}
        
	\section{Requisiti non Funzionali}
	Il primo vicolo individuato è quello economico. Il costo del servizio, essendo l'attività non a scopo di lucro e mantenuta grazie all'azione dei volontari, deve essere minimo. Questo è comprensivo dell'istallazione, dei materiali e dei costi di servizio.
	Un secondo vincolo rappresenta la sicurezza, l'eventuale introduzione di strumentazione all'interno del canile non deve rappresentare in alcun modo un pericolo per la salute dell'animale.
	
    Il sistema dovrà rispettare alcuni requisiti non funzionali che ne determineranno la \textbf{qualità}:
        \subsection{Di Sistema}
            \begin{itemize}
            \item \textbf{Reattività}: 
            \begin{itemize}
                \item l'utente non deve percepire \textbf{ritardi} nell'ordine dei secondi tra l'invio di un comando e l'esecuzione dello stesso all'interno della piattaforma. 
                \item le notifiche standard del sistema devono essere mostrate ai relativi utenti con un ritardo complessivo massimo non superiore al minuto.
                \item le notifiche urgenti del sistema, ossia quelle relative a malfunzionamenti gravi o alla salute dell'animale, devono essere mostrate ai relativi utenti con un ritardo complessivo massimo non superiore ai dieci secondi.
            \end{itemize}
            
            \item \textbf{Scalabilità}: L'applicativo deve necessariamente consentire di aumentare o diminuire il numero di animali gestiti. Ciò deve avvenire senza una sensibile ripercussione sulle prestazioni del sistema e un disagio minimo a livello pratico. Per questo motivo, non devono essere presenti ulteriori connessioni cablate al di fuori dell'alimentazione già presente. 
            
            \item \textbf{Fault tolerance}: la \textbf{gestione degli errori} deve essere adeguatamente implementata affinché le interruzioni involontarie non danneggino innanzitutto la salute degli animali.
            Eventuali malfunzionamenti di apparecchiature o sensoristica all'interno del canile non devono pregiudicare il funzionamento complessivo dell'applicativo, ma al massimo della singola unità logica. 
    \end{itemize}
        
    \subsection{Della Sensoristica}
    \begin{itemize}
        \item \textbf{Precisione}:
            \begin{itemize}
                \item \textbf{Bilancia} lo scostamento massimo tra più misure deve essere inferiore o uguale a 10 grammi.
                \item \textbf{Temperatura} lo scostamento massimo tra più misure deve essere inferiore o uguale a 2 gradi.
                \item \textbf{Umidità} lo scostamento massimo tra più misure deve essere inferiore o uguale a 5\%.
                \item \textbf{Livello acqua} lo scostamento massimo tra più misure deve essere inferiore o uguale a 1 centimetro.
                \item \textbf{Frequenza cardiaca} lo scarto tra misure sulla stessa frequenza deve essere al massimo 10 battiti.
            \end{itemize}
        \item \textbf{Risoluzione}:
            \begin{itemize}
                \item \textbf{Bilancia} il range di misura, partendo da vuota, deve riuscire a comprendere almeno un Kg.
                \item \textbf{Temperatura} il range deve variare tra 0 e 50 gradi. %considerato che i cani piccoli crepano a -6 e i grandi a -12, ma tanto a 0 gradi sono tutti in temperatura critica direi che ci sta
                \item \textbf{Umidità} il range di umidità deve essere incluso tra 20-80\%.
                \item \textbf{Livello acqua} Il range deve variare almeno di tre cm. 
                \item \textbf{Videocamera}: necessaria almeno una risoluzione di 420p, non è richiesta la visione notturna. 
                \item \textbf{Frequenza cardiaca} la misurazione deve essere in grado di rilevare una frequenza che oscilla tra i 60 e i 160 battiti al minuto. 
                \item \textbf{Microfono} deve essere in grado di determinare suoni ad elevata intensità compresi tra 50 e 130 dB. 
            \end{itemize}
        \item \textbf{Accuratezza}:
            \begin{itemize}
                \item \textbf{Bilancia} lo scostamento massimo dal valore reale deve essere inferiore o uguale a 10 grammi.
                \item \textbf{Temperatura}  Lo scostamento dalla temperatura effettiva deve essere minore di 2 gradi.
                \item \textbf{Umidità} Si accetta un massimo di 5\% accuratezza.
                \item \textbf{Livello acqua} l'accuratezza minima deve essere di 2cm.
                \item \textbf{Frequenza cardiaca} l'errore non deve superare i cinque battiti al minuto.
            \end{itemize}
    \end{itemize}    
    

	\section{Requisiti Implementativi}
