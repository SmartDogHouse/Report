\chapter{Aspetti di Domain Driven Design}
    \section{Individuazione del Core Domain}
	All'interno del Knowledge Krunching, domande significative sono state fatte per capire i concetti fondamentali del dominio.
	Non tutte le parti hanno egual importanza, in questo capitolo si cerca individuare il core domain. Infatti, è necessario ridurre la complessità focalizzandosi su quelle più importanti, ciò aiuta a diminuire la difficoltà globale.
	Inoltre c'è da specificare che il core domain non va confuso con l’organizzazione dell'azienda, nel nostro caso il software va a risolvere un problema mirato, non coincidente con l'intero ambito aziendale.
	Ulteriore attenzione è stata fatta al fatto che il core possa cambiare con il tempo. 
    \section{Identificazione dei Subdomain}
    \section{Definizione dei Bounded Context}
    \section{Definizione delle Context Map}