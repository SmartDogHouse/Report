% -*- root: ../main.tex -*-
\chapter{Design Architetturale}
    \section{Architettura Generale}
    \section{Pattern Architetturali Utilizzati}
    \section{Scelte Tecnologiche Cruciali}
        \subsection{Microcontrollori}
        Per quanto riguarda l'implementazione del codice nel microcontrollore, sono state presi in considerazione quattro scelte relative ai framework di sviluppo e ai linguaggi da adottare: 
        \begin{itemize}
            \item \textbf{C} è il linguaggio attualmente più diffuso nell'ambito. 
            
            \item \textbf{Node-RED} è un tool di programmazione flow-based basato su browser, sviluppato su NodeJs e programmabile in JavaScript.
            
            \item \textbf{MicroPython} è il sub domain relativo alla gestione e monitoraggio video del canile.
            
            \item \textbf{Espruino} è il sub domain relativo alla gestione e monitoraggio video del canile.
        \end{itemize}
        Sicuramente il più performante e con più librerie è C, ma risulta anche difficilmente testabile e di difficile comprensione ai non esperti. Node-RED, basato su Node.js, ha il pieno vantaggio del suo modello non bloccante e event-driven. MicroPython d'altra parte offre un modello ad oggetti, testabile, sia semplice che elastico, con una buona base di librerie. Infine, Espruino, interprete JavaScript anch'esso, di facile utilizzo e dall'interfaccia semplificata. 
        I fattori chiave che il team ha preso in considerazione per la scelta sono stati: 
        \begin{itemize}
            \item la potenza espressiva rapportata alla chiarezza del linguaggio. 
            \item la testabilità del codice
            \item il numero di libsrerie e la futura possibilità di estensione del progetto.
        \end{itemize}
        Alla luce delle considerazioni fatte, si è scelto MicroPython per lo sviluppo, il testing e l'automatizzazione.
