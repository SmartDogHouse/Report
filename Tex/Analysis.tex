% -*- root: ../main.tex -*-
\chapter{Analisi dello spazio del problema}
	\section{Knowledge crunching}	
	\section{Definizione Ubiquitus language}	
	L'Ubiquitous Language deve essere espresso nel modello di dominio, infatti  unisce le persone del team di progetto.
    Lo scopo è eliminare le imprecisioni e le contraddizioni degli esperti di dominio, non è infatti imposto da questi, ma raggiunto collaborativamente.
    L'Ubiquitous Language si evolve nel tempo, non è definito interamente in una sola riunione, i concetti spesso si aggiungono, vengono sviscerati e partecipano nella comprensione del dominio. Infatti quelli che non fanno parte dell'Ubiquitous Language devono essere rifiutati.





\centering
\captionsetup{labelformat=empty}
\label{tab:UL}
\resizebox{\linewidth}{!}{%
\begin{tabular}{|>{\hspace{0pt}}m{0.35\linewidth}|>{\hspace{0pt}}m{0.28\linewidth}|>{\hspace{0pt}}m{0.61\linewidth}|} 
\hline
Termine & Equivalenza & Descrizione \\ 
\hline
Gestore & Manager A & Persona che ha in carico il canile \\ 
\hline
Veterinario & Vet A & Persona incaricata alla salute dei cani qualora vi siano urgenze \\ 
\hline
Addetto & Attendant & Persona pagata per svolgere mansioni manuali, con orari e presenza precisa \\ 
\hline
Addetto al rifornimento & Food attendant & Addetti che tra le mansioni hanno quella di rifornire il cibo nelle gabbie \\ 
\hline
Volontario & Volunteer & Persona non pagata per svolgere mansioni, prese in carico con presenza incostante \\ 
\hline
Sorvegliante & Supervisor & Addetto che si occupa della videosorveglianza \\ 
\hline
Degente(?) & Patient & Animale che sta lì perchè malato e viene liberato una volta curato \\ 
\hline
In terapia & Curing & Animale che prende farmaci o necessita di alimentazione particolare. E' seguito dal vet \\ 
\hline
In osservazione & Under observation & Animale malato, infortunato o gravido che necessita monitoraggio con collare smart \\ 
\hline
Taglia & Size & Dimensione dell'animale, da cui dipende cibo \\ 
\hline
Temperatura corporea & Body temperature & Temperatura misurata all'altezza del collo (NON rettale) colon \\ 
\hline
Parametri vitali & Vital signs & battiti, Temperatura,ecc \\ 
\hline
Parametro (vitale) anomalo & Abnormal parameter & Parametro che si trova al di fuori del range stabilito \\ 
\hline
Anomalia & Abnormality & Dato riscontrato che diffrisce dal solito (solito: definito dallo storico) \\ 
\hline
Battito cardiaco & heartbeat & battito misurato all'altezza del collo \\
\hline
\end{tabular}
}


\chapter{Analisi dei Requisiti}
	\section{Requisiti di Business}
	
	\section{Requisiti Utente}
        \subsection{User stories}
	    \subsection{Casi d'uso}
	\section{Requisiti Funzionali}
	\section{Requisiti non Funzionali}
	\section{Requisiti Implementativi}
