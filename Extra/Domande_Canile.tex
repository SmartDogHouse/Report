\documentclass{report}
\usepackage[utf8]{inputenc}
\usepackage[T1]{fontenc}
\usepackage{longtable}
\usepackage{graphicx}
 \usepackage{array}
 \usepackage{caption}
 \usepackage{graphicx}
 \usepackage{rotating}
 \usepackage[usenames,dvipsnames]{xcolor}
\usepackage{tcolorbox}
\usepackage{tabularx}
\usepackage{array}
\usepackage{colortbl}
\usepackage{color}
\usepackage{multicol}
\tcbuselibrary{skins}
%\usepackage[italian]{babel}

%Disable all warnings issued by latex starting with "You have..."
\usepackage{silence}
\WarningFilter{latex}{You have requested package}
\pdfsuppresswarningpagegroup=1

%Bib
\usepackage[
backend=biber,
style=alphabetic,
sorting=ynt
]{biblatex}
\addbibresource{References.bib}

\usepackage{csquotes}
%\usepackage{natbib}


%Import

\usepackage{tabularx}
\usepackage{marvosym}
\usepackage{fancyvrb}
%\usepackage[usenames]{color}
\usepackage[hidelinks]{hyperref}
\usepackage{url}
\usepackage{graphicx}
\usepackage{xcolor}
\usepackage{amsmath,amsfonts,amssymb,amsthm,mathtools}
\usepackage{caption}
\usepackage{enumerate}
\usepackage{multicol}
\usepackage{subcaption}
\usepackage{float}
\usepackage{indentfirst}
\usepackage{listings}
\usepackage{tocloft}
\usepackage{ifthen}
%\usepackage[most]{tcolorbox}
\usepackage{pgfplots}
%\usepackage{Style/pgfplotsthemetol}

\pgfplotsset{compat=1.16}
\definecolor{lstgrey}{rgb}{0.94,0.95,1}
\lstset{
    language=python,
    backgroundcolor=\color{lstgrey},
    frame=single,
    rulecolor=\color{lstgrey}, % make frame "invisible"
    captionpos=t,
    tabsize=2,
    numberbychapter=false,
    showstringspaces=false,
    basicstyle=\footnotesize,
    breaklines=true,
}
%LINK CLICCABILI
\hypersetup{
    colorlinks = true,
    linkcolor = .,
    citecolor = {blue},
    linkbordercolor = {white},
    urlcolor = {blue},
}
%TABELLE TBCOLORBOX
\tcbset{
        enhanced,
        colback=red!5!white,
        boxrule=0.1pt,
        colframe=red!75!black,
        fonttitle=\bfseries
       }
       
%TAB2
\newcolumntype{Y}{>{\raggedleft\arraybackslash}X}

\tcbset{tab1/.style={fonttitle=\bfseries\large,fontupper=\normalsize\sffamily,
colback=yellow!10!white,colframe=red!75!black,colbacktitle=Salmon!40!white,
coltitle=black,center title,freelance,frame code={
\foreach \n in {north east,north west,south east,south west}
{\path [fill=red!75!black] (interior.\n) circle (3mm); };},}}

\tcbset{tab2/.style={enhanced,fonttitle=\bfseries,fontupper=\normalsize\sffamily,
colback=yellow!10!white,colframe=red!50!black,colbacktitle=Salmon!40!white,
coltitle=black,center title}}

%PATH IMMAGINI
%\graphicspath{{Images/}{DrawIo/}}
\newcommand{\emailaddr}[1]{\href{mailto:#1}{\texttt{#1}}}



\title{\LARGE
        DOMANDE CANILI DI CESENA e PISA  \\
        \hrulefill \\
        Smart DogHouse \\
        \hrulefill \\
    }

    \author{
        Kiade Sara \\ \emailaddr{sara.kiade@studio.unibo.it}
        \and 
        Caminati Gyordan \\ \emailaddr{gyordan.caminati@studio.unibo.it} 
        \and
        Lirussi Igor \\ \emailaddr{igor.lirussi@studio.unibo.it}
        \\ \\ \\ 
    }

\begin{document}
\maketitle


\begin{enumerate}
\def\labelenumi{\arabic{enumi}.}
\item
  Di cosa vi occupate?

  \begin{itemize}
  \tightlist
  \item
    \textbf{Cesena:} ``ci occupiamo di gestire segnalazioni riguardanti
    cani randagi/abbandonati che ci arrivano dai vigili o dalla
    cittadinanza. Una volta recuperato il cane cerchiamo di ritrovare la
    famiglia a cui appartiene, se la ricerca non va a buon fine il cane
    viene ospitato in canile. Trascorso un lasso di tempo in cui al
    proprietario viene concesso di recuperare il cane, iniziamo la
    ricerca di una nuova famiglia che lo ospiti. Per facilitare
    l'incontro con una nuova famiglia, inoltre, al cane viene impartito
    un addestramento di base che possa abituarlo, specie se in passato
    veniva maltrattato, a una positiva interazione con le persone. Non
    ci occupiamo di altri animali''.
  \item
    \textbf{Pisa: }``il nostro lavoro consiste nel recupero di cani e
    gatti. I gatti vengono ospitati in canile solo quando hanno bisogno
    di cure. Una volta che le condizioni di salute del gatto si sono
    stabilizzate, esso viene rilasciato. I cani rimangono con noi fino
    all'adozione. La nostra area di competenza si estende fino a Livorno
    e ai comuni limitrofi. Se un cane recuperato non ha un
    identificativo, esso viene chippato e, in caso il proprietario venga
    a recuperarlo paga una penale. Ci arrivano anche cani sequestrati
    per maltrattamento o allevamento abusivo''.
  \end{itemize}
\item
  Come è strutturata la vostra attività?

  \begin{itemize}
  \tightlist
  \item
    \textbf{Cesena:} ``Ricevuta una segnalazione cerchiamo di capire di
    che cane si tratta, teniamo un archivio cartaceo con i cani che ci è
    già capitato di recuperare in maniera tale da poterlo individuare
    più facilmente incrociando zona e caratteristiche estetiche. Molte
    volte si tratta di segnalazioni affrettate; per questo motivo ci
    accertiamo della necessità del nostro intervento prima di muoverci,
    anche perché tutte le volte che dobbiamo spostarci per recuperare un
    cane con proprietario, quest'ultimo si trova a dover pagare una
    penale. Recuperato un cane, questo viene chippato e custodito, viene
    addestrato e gli vengono fornite tutte le cure di cui ha bisogno,
    fino all'adozione. Ai cani assegniamo un bollino colorato in base
    alla pericolosità: i cani contrassegnati con un bollino verde sono
    mansueti e abituati al contatto con le persone, quelli con bollino
    rosso sono estremamente aggressivi e spesso hanno alle spalle una
    lunga storia di maltrattamenti; il bollino arancione, invece,
    contrassegna i cani di media aggressività. In base al colore del
    bollino cambia l'insieme delle persone a cui è permesso interagire
    con quel cane. Manteniamo dei registri cartacei con tutti i dati dei
    cani che ospitiamo, da quelli anagrafici a quelli sanitari''.
  \item
    \textbf{Pisa: }``Riceviamo telefonate in qualsiasi orario di persone
    che ci segnalano l'avvistamento di un cane randagio. In realtà i
    cani randagi non esistono praticamente più e al 98\% si tratta di
    cani abbandonati o scappati. La prima cosa che facciamo è recarci
    sul luogo dell'avvistamento e cercare di capire chi è il
    proprietario, guardando soprattutto nelle zone vicine al luogo del
    ritrovamento. Spesso ci capita che a fare la segnalazione siano i
    vicini che, pur sapendo esattamente di chi è il cane scappato dal
    cancello, si rivolgono a noi dicendo di aver avvisato un randagio
    per dispetto. Recuperato il cane si procede alla lettura del chip,
    se è presente. In molti casi sono sufficienti queste fasi per
    ritrovare il proprietario, in altri il cane viene recuperato e
    portato in canile. Vengono messi degli annunci sui social per
    ritrovare il proprietario, ma, una volta in canile, capita che il
    proprietario non lo venga più a recuperare perché non è disposto a
    pagare la penale per l'assenza del chip. Ci è capitato molte volte
    che, pur di non dover pagare, il proprietario di un cane venisse in
    canile fingendosi un normale cittadino interessato ad un'adozione e
    facendo ricadere casualmente la scelta proprio sul cane che aveva
    perso. Una volta portato il cane in canile lo facciamo visitare dal
    veterinario che ci dice se è sano o se ha bisogno di particolari
    attenzioni, ad esempio per quanto riguarda il regime alimentare;
    valutiamo il suo comportamento e decidiamo in quale gabbia
    collocarlo, se da solo o in compagnia. ''.
  \end{itemize}
\item
  Quali sono i ruoli e le figure all'interno del canile?

  \begin{itemize}
  \tightlist
  \item
    \textbf{Cesena:} ``Per quanto riguarda il personale pagato non ci
    sono separazioni particolari tra le figure. Tutti si occupano più o
    meno di tutto. Siamo tutti addestratori e ci occupiamo tutti del
    recupero dei cani e della burocrazia che c'è dietro. Alcuni sono più
    bravi in una cosa piuttosto che in un'altra e quindi determinati
    incarichi gli vengono assegnati più spesso, ma sulla carta siamo
    tutti allo stesso livello. Per quanto riguarda i volontari, invece,
    ognuno di loro si deve sempre rivolgere a un dipendente che
    rappresenta una figura di riferimento. I volontari hanno delle
    limitazioni, ad esempio non possono occuparsi di dare il cibo ai
    cani, portarli fuori o interagire con quelli che non hanno il
    bollino verde senza l'assistenza o il permesso di uno degli
    addestratori''.
  \item
    \textbf{Pisa: }``Le figure principali sono: il gestore, che si
    occupa interamente della gestione del canile, dalle mansioni più
    umili a tutto ciò che riguarda l'aspetto burocratico, le spese e i
    consumi; gli addetti, che si occupano di tutte le mansioni di
    routine del canile come dare il cibo ai cani, pulire le gabbie,
    portarli nello sgambatoio, gestire le persone che arrivano per
    adottare; e i volontari che si occupano più che altro di giocare con
    i cani, portarli fuori dalla gabbia e dargli affetto. Capita però
    che in situazioni movimentate, in cui il carico di lavoro degli
    addetti risulta essere eccessivo i volontari diano una mano anche
    nelle altre mansioni, tipo la pulizia delle gabbie.''
  \end{itemize}
\item
  Quali sono i collaboratori esterni?

  \begin{itemize}
  \tightlist
  \item
    \textbf{Cesena:} ``Gestiamo interamente tutto ciò che riguarda il
    canile, anche a livello sanitario, perciò non abbiamo collaboratori
    esterni. Capita a volte di doverci rivolgere a una clinica per
    operazioni chirurgiche particolarmente complesse, ma è molto raro
    poiché il nostro responsabile sanitario riesce ad occuparsi
    praticamente di tutto''.
  \item
    \textbf{Pisa: }``Come collaboratori esterni potremmo contare i
    volontari, visto che non sono dipendenti diretti del canile e la
    loro attività non è regolare e i veterinari. All'interno del canile
    abbiamo un responsabile sanitario, che coordina dei veterinari che
    hanno uno studio e un'attività loro ma si rendono disponibili anche
    per il canile.''
  \end{itemize}
\item
  Avete dei volontari? Che mansioni svolgono?

  \begin{itemize}
  \tightlist
  \item
    \textbf{Cesena:} ``Sì. Imparano dai dipendenti come si gestisce un
    cane, imparano nozioni sull'addestramento, sull'alimentazione e sui
    modi corretti per giocare. Ci aiutano nelle mansioni, anche se
    sempre sorvegliati e ci permettono di fornire ai cani un ricambio di
    persone evitando così che i cani si affezionino solo ai dipendenti e
    che si comportino in maniera socievole solo con loro, non essendo in
    grado di replicare lo stesso comportamento anche con altre persone.
    ''.
  \item
    \textbf{Pisa: }``Sì; i volontari sono una risorsa, tant'è che
    durante il lockdown il canile ha risentito parecchio della loro
    assenza. La loro mansione principale è quella di creare un rapporto
    affettivo con i cani che ospitiamo, ma ci danno una mano anche nelle
    altre mansioni quando ne abbiamo bisogno''
  \end{itemize}
\item
  Dove si colloca la figura del veterinario?

  \begin{itemize}
  \tightlist
  \item
    \textbf{Cesena:} ``Abbiamo un unico responsabile sanitario
    all'interno del canile che si occupa della salute di tutti i cani e
    svolge dalle più piccole mansioni come lo sverminamento alle
    operazioni chirurgiche anche di medio-alta complessità''.
  \item
    \textbf{Pisa: }``Il veterinario principale è il responsabile
    sanitario che ha lo studio all'interno del canile; esso coordina un
    insieme di veterinari che hanno uno studio proprio e clienti propri
    ma che decidono di offrire il loro servizio anche al canile.''
  \end{itemize}
\item
  Quanto spesso il veterinario visita i cani?

  \begin{itemize}
  \tightlist
  \item
    \textbf{Cesena:} ``Dipende dalla necessità. Se non succede niente fa
    un giro di ricognizione un paio di volte a settimana; se invece uno
    degli addetti nota un comportamento strano in un cane (ad esempio si
    gratta spesso l'orecchio), il nostro veterinario lo visita e ne
    approfitta per fare un giro anche degli altri cani. Una visita
    approfondita, invece viene fatta con una regolarità diversa a
    seconda del cane.''.
  \item
    \textbf{Pisa: }``I cani vengono visitati in caso in cui manifestino
    un malessere. Periodicamente un veterinario esegue interventi di
    routine come le vaccinazioni, lo sverminamento, il trattamento anti
    pulci ecc..''
  \end{itemize}
\item
  Cosa avrebbe senso automatizzare? A parte distribuzione cibo e acqua e
  sorveglianza?

  \begin{itemize}
  \tightlist
  \item
    \textbf{Cesena:} ``Sarebbe bello se si riuscisse ad avere un sistema
    che riesca a capire se un cane ha davvero ingerito la pillola che
    gli viene mischiata al cibo e che non l'abbia ad esempio sputata.
    Sarebbe inoltre utile sapere quando un cane in gravidanza va in
    travaglio. Sappiamo che per i cavalli viene inserito un dispositivo
    che quando si rompono le acque si spezza e invia un segnale. Non è
    un dispositivo adatto ai cani, ma ci chiedevamo se si potesse
    implementare un sistema simile. Sarebbe utile anche solo una
    videocamera che ci permetta di vedere il cane in gravidanza da casa,
    evitandoci di dover fare numerosi sopralluoghi notturni fino alla
    data del parto.''
  \item
    \textbf{Pisa: }``Non penso ci siano altre cose da poter
    automatizzare, a parte magari la gestione dei turni del personale.
    Il contatto umano è molto importante per i cani, pertanto anche solo
    entrare nella gabbia di un cane per svolgere delle mansioni attiva
    degli aspetti comportamentali del cane e può fungere da mezzo
    educativo''
  \end{itemize}
\item
  Avete una o più figure preposte alla sorveglianza?

  \begin{itemize}
  \tightlist
  \item
    \textbf{Cesena:} ``No, però capita di dover fare dei turni di notte
    o dei sopralluoghi per controllare la situazione''
  \item
    \textbf{Pisa: }``No, non ci sono addetti con questa specifica
    mansione, per una questione di fondi. In realtà sarebbe utile perché
    ci è già capitato che qualcuno provasse a entrare di nascosto''
  \end{itemize}
\item
  Avete un sistema di videosorveglianza?

  \begin{itemize}
  \tightlist
  \item
    \textbf{Cesena:} ``No, nessuno, ma stiamo pensando di installare
    qualche telecamera per controllare le condizioni dei cani da remoto,
    appena avremo la liquidità per farlo.''
  \item
    \textbf{Pisa: }``Attualmente non disponiamo di un sistema di
    videosorveglianza, abbiamo le inferriate e teniamo le gabbie chiuse
    e abbiamo un allarme.''
  \end{itemize}
\item
  Avete un sistema informatico?

  \begin{itemize}
  \tightlist
  \item
    \textbf{Cesena:} ``No, ma ne sentiamo fortemente la mancanza. Un
    sistema che ci permettesse di facilitare la gestione degli aspetti
    burocratici e di mantenere registri digitali con le informazioni dei
    nostri cani ci risparmierebbe molto lavoro. Attualmente l'unico
    supporto che abbiamo è l'anagrafe canina, ma il sistema permette di
    fare poche cose, a partire dalla ricerca di un cane''
  \item
    \textbf{Pisa: }``Utilizziamo il computer solo per leggere i chip e
    consultare l'anagrafe canina, tutto il resto viene gestito in
    maniera cartacea''
  \end{itemize}
\item
  Avete un vostro gestionale?

  \begin{itemize}
  \tightlist
  \item
    \textbf{Cesena:} ``No''
  \item
    \textbf{Pisa: }``No''
  \end{itemize}
\item
  Una connessione internet? Ethernet/Wi-Fi?

  \begin{itemize}
  \tightlist
  \item
    \textbf{Cesena:} ``Utilizziamo la connessione che ci fornisce il
    comune, ma ha molte limitazioni. Ci sarebbe utile mettere su
    qualcosa di più performante''
  \item
    \textbf{Pisa: }``Solo connessione via cavo''
  \end{itemize}
\end{enumerate}

\begin{quote}
\end{quote}

\begin{enumerate}
\def\labelenumi{\arabic{enumi}.}
\item
  Chi si occupa della distribuzione del cibo? Come funziona? Avete dei
  grossi contenitori che fanno da serbatoio di cibo, oppure riempite la
  ciotola volta per volta?

  \begin{itemize}
  \tightlist
  \item
    \textbf{Cesena:} ``Porgiamo direttamente la ciotola con il cibo al
    cane, continuando a tenerla in mano. Terminato il pasto la togliamo,
    senza lasciarla a disposizione del cane. In questo modo il cane non
    diventa territoriale nei confronti del cibo e rimarchiamo la nostra
    posizione di `alpha'. Per quanto riguarda la dieta abbiamo diversi
    contenitori con crocchette di diverso tipo e le diamo ai cani in
    base alla taglia, alle condizioni di salute e alle condizioni
    meteorologiche''
  \item
    \textbf{Pisa: }``Le ciotole vengono riempite dagli addetti, mentre
    l'acqua rimane sempre a disposizione. Al termine dei pasti le
    ciotole vengono lavate e rimesse al loro posto. Le crocchette sono
    le stesse per tutti, variano solo in base alla taglia o se il cane
    non può mangiare quelle standard''
  \end{itemize}
\item
  Come funziona la distribuzione dell'acqua? Avete tipo degli
  abbeveratoi o riempite la ciotola e se quello la rovescia di notte
  pace?

  \begin{itemize}
  \tightlist
  \item
    \textbf{Cesena:} ``La ciotola dell'acqua è fissata su un supporto
    che non permette al cane di rovesciarla. L'acqua rimboccata quando
    il cane la consuma e, a fine giornata, le ciotole vengono lavate e
    riempite nuovamente per evitare che facciano `il verde', cosa che
    rientra tra i principali segnali di maltrattamento''
  \item
    \textbf{Pisa: }``Le ciotole vengono riempite dagli addetti, mentre
    l'acqua rimane sempre a disposizione. Al termine dei pasti le
    ciotole vengono lavate e rimesse al loro posto. Le crocchette sono
    le stesse per tutti, variano solo in base alla taglia o se il cane
    non può mangiare quelle standard''
  \end{itemize}
\item
  Avreste la possibilità di attaccarci un sistema di distribuzione di
  acqua ad ogni gabbia?

  \begin{itemize}
  \tightlist
  \item
  \end{itemize}
\item
  Monitorate il consumo di cibo e acqua dei cani? Avrebbe senso farlo?

  \begin{itemize}
  \tightlist
  \item
    \textbf{Cesena:} ``Non abbiamo modo di monitorarli in maniera
    precisa, ci limitiamo a vedere ad occhio qual è la situazione.
    Avrebbe molto senso se si riuscisse a fare perché sarebbe un valido
    aiuto per il veterinario e ci permetterebbe di intercettare disturbi
    alimentari che possono essere sintomo di un malessere anche grave.
    In quasti casi è importante accorgersene per tempo e, con tanti cani
    da gestire, questo risulta essere più difficile.
  \item
    \textbf{Pisa: }``Non monitoriamo i consumi di cibo e acqua ma sembra
    essere un'idea interessante.''
  \end{itemize}
\item
  È interessato ad avere notifiche sui parametri vitali dei cani?

  \begin{itemize}
  \tightlist
  \item
    \textbf{Cesena:} ``Sicuramente! Sarebbe davvero un valido aiuto e ci
    permetterebbe di avere molto più controllo sullo stato di salute dei
    cani. Se si riuscisse a fare davvero, con costi contenuti,
    porterebbe sicuramente una grande innovazione.''
  \item
    \textbf{Pisa: }``Anche questa sembra essere un'idea molto
    interessante, più dell'automatizzazione di acqua e cibo che, per
    quanto utile, si riesce già a gestire. Avere il controllo dei
    parametri principali sarebbe un'ottima cosa, anche se sarebbe bello
    monitorare anche altri parametri come l'attività fisica e la
    pressione sanguigna''
  \end{itemize}
\end{enumerate}

\begin{enumerate}
\def\labelenumi{\arabic{enumi}.}
\item
  Avete cani con esigenze particolari? Incinta, con malattie, con un
  regime alimentare particolare?

  \begin{itemize}
  \tightlist
  \item
    \textbf{Cesena:} ``Sì, abbiamo cani con patologie più o meno gravi,
    soprattutto quando cominciano ad essere anziani. Attualmente, ad
    esempio, ospitiamo Chewbecca, un cane che soffre di leishmaniosi e
    che pertanto necessita di cure specifiche e un regime alimentare
    particolare.''
  \item
    \textbf{Pisa: }``Sì, ce ne sono diversi. Molti purtroppo vengono
    abbandonati anche per questo motivo. Cerchiamo di prendercene cura
    come possiamo, a volte anche tramite raccolte fondi.''
  \end{itemize}
\item
  Siete interessati al rilevamento dei parametri ambientali del canile?
  temperatura, umidità, rumore\ldots{}

  \begin{itemize}
  \tightlist
  \item
    \textbf{Cesena:} ``Non molto. Anche se a volte ci basiamo su
    temperatura o umidità per la scelta delle crocchette da dare (ad
    esempio di inverno diamo crocchette più proteiche), in linea di
    massima non ci interessa avere questo tipo di rilevazioni perché è
    facile che i valori siano facilmente influenzabili da mille fattori.
    Se anche ci fosse la possibilità di conoscere queste rilevazioni non
    credo che ci guarderemmo molto, preferendo fidarci della nostra
    percezione ed esperienza. ''
  \item
    \textbf{Pisa: }``Sicuramente è una cosa in più, ma non vedo che
    utilità possa avere nel concreto. Non è una pessima idea, ma non è
    neanche la più utile.''
  \end{itemize}
\end{enumerate}



\end{document}
